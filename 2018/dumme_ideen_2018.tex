% Template inspired by Micheal Müller's Book Template:
% https://www.overleaf.com/latex/templates/simple-a5-book-template/gkryymktcyzs

\documentclass[a5paper,pagesize,10pt,bibtotoc,pointlessnumbers,normalheadings,DIV=9,twoside=false]{scrbook}

\KOMAoptions{DIV=last}
\RedeclareSectionCommand[beforeskip=0pt,afterskip=5mm]{chapter}
\addtokomafont{chapter}{\large}

%\usepackage{trajan}
\usepackage{tgpagella}
\usepackage[T1]{fontenc}
\usepackage[utf8]{inputenc}
\usepackage[ngerman]{babel}
\usepackage{fontspec}

%\usepackage[babel,german=guillemets]{csquotes}

%\usepackage[sc]{mathpazo}
%\linespread{1.05} 

\usepackage{verbatim}
\usepackage{listings}

\usepackage{titlesec}
\usepackage{authblk}

\usepackage{setspace}

\newcommand{\attribution}[1]{
    \textbf{Inspiration:} #1
}

%\setmainfont{TeX Gyre Pagella}

%\hyphenation{un-ver-packt}

%\setlength{\parindent}{10pt}
%\setlength{\parskip}{1.4ex plus 0.35ex minus 0.3ex}
%\setlength{\parskip}{1.4ex plus 0.35ex minus 0.3ex}

\title{Dumme Ideen 2018}   
\author{Andreas Schmid} 
\date{\today} 

\begin{document}

% Title page

\maketitle

% Inspirational quote thingy
\newpage{}
\thispagestyle {empty}

\vspace*{2cm}

\begin{center}
	\Large{\parbox{10cm}{
		\begin{raggedright}
		{\Large 
			\textit{Ein Buch ohne inpirierendes Zitat am Anfang lohnt sich nicht zu lesen.}
		}
	
		\vspace{.5cm}\hfill{---Andreas Schmid}
		\end{raggedright}
	}
}
\end{center}

\newpage

% toc

\singlespacing
\tableofcontents 

% Content!
\onehalfspacing

\chapter*{Vorwort}
Ich habe viele Ideen.
Seit 2018 schreibe ich sie auch auf.
Einige davon sind lustig, einige nützlich, einige peinlich - aber eins haben die meisten gemeinsam: Ich werde sie niemals umsetzen.
Deshalb hier eine stetig wachsende Sammlung von Ideen, macht damit was ihr wollt.
Aber wenn ihr eine davon umsetzt, gebt mir bitte Bescheid - das würde mich sehr freuen!

%\chapter{Rotes Telephon zwischen FIL und TechBase}
%\chapter{real live 3rd person steuerung (drohne + vr-brille)}
%\chapter{tag-basierter musikrecommender}
\chapter{Kochsendung mit Werkzeug}

Man kennt sie, man liebt sie - Kochsendungen im Fernsehen oder im Internet.
Doch effektiv ist es doch immer das gleiche:
Ein leicht adipöser Herr mittleren Alters steht in einer viel zu gut ausgestatteten Küche, alle Zutaten sind in einer eigenen Schüssel, sodass man sich fragt, wer das ganze Zeug am Ende abspült und es wird zum hundertsten Mal irgend ein Bratensatz mit Rotwein abgelöscht, bevor wie von Zauberhand das bereits vorbereitete, fertige Gericht aus dem Ofen geholt wird.
Und auf keinen Fall das Lorbeerblatt vergessen!

Eine junge, von Killerspielen und Avengers-Filmen geprägte Zielgruppe spricht dieses Format wohl kaum an - was fehlt sind Action, Explosionen und Gefahr.
Deshalb die Idee: Warum nicht eine klassische Kochsendung etwas spannender gestalten, indem statt Kochmesser und Pürierstab Kreissäge und Bohrmaschine zum Einsatz kommen?
Dazu noch ein harter Heavy Metal-Sound"-track aus dem Makita-Baustellenradio und ein tätowierter Moderator, der ständig rumschreit und fertig ist die Kochsendung für die nächste Generation!

\chapter{Kochsendung mit Campingkocher}

Die Kochsendung mit Campingkocher stellt ein krasses Gegenstück zur letzten Idee dar.
Ein minimalistischer und auf das Nötigste beschränkter Lebensstil ist für die nachhaltige Hipster-Ge"-ne"-ra"-tion so wichtig wie der Fahrradparkplatz vor dem Un"-ver"-packt-Laden.
Eine kleine Japanerin zeigt uns, auf was wir eigentlich alles verzichten können, um ein glückliches Leben frei von Konsum und Kapitalismus zu führen.
Außerdem passt dann unser ganzes Zeug auch gut in den Dakine-Rucksack für den nächsten inspirierenden Trip nach Bali.

Da beim Backpacken das Kochen natürlich zu einer etwas größeren Herausforderung wird und wir uns nicht sicher sind, wie lange die gebratenen Nudeln beim Streetfood-Stand dieser älteren Dame nun wirklich schon im Wok vor sich hin brodeln, sind wir oft auf den Campingkocher angewiesen.
Dosenravioli sind natürlich auch nicht das wahre (Weißblech! Pfui!), deshalb brauchen wir eine Kochsendung, die nur mit dem Nötigsten auskommt:
Ein Campingkocher, ein Messer und alles, was uns Mutter Natur so bietet.
Und natürlich ein Hippie-Moderator mit beruhigender stimme, der uns zeigt, wie man leckere Gerichte mit dem Campingkocher zubereitet - damit wir auch auf Reisen nicht auf Nudeln mit Pesto verzichten müssen.
Amazing!

%\chapter{1. April: Besen anzünden}
%\chapter{Tutorialvideo: Besen bauen}
%\chapter{Wetteinsatz: Mit dem Radl aufs Summerbreeze fahren}
\chapter{Komplettes Videospiel als QR-code}
\attribution{Florian Bockes}

QR-Codes werden normalerweise verwendet, um Links zu Webseiten auf Papier zu drucken, Gegenstände mit IDs zu versehen und so weiter.
Aber in einem QR-Code kann man durchaus ein bisschen was an Daten unterbringen - 3 Kilobyte bei einem klassischen QR-Code, um genau zu sein.
Das ist zwar nicht wahnsinnig viel, aber vielleicht kann man ja mehrere kombinieren, um ein kleines Videospiel unterzubringen.
Vielleicht ein Textadventure?

Man würde dann einfach den QR-Code mit dem Smartphone scannen und schon beginnt das Spiel.
Ideal zum Beispiel für ein Wartezimmer, eine Schnitzeljagd, oder um zwischen den Spielen eine Rickroll zu verstecken.

%\chapter{Kalender mit falsch zugeordneten Zitaten}
% TODO: Abbildung

\chapter{Fotos von allen Kontakten am Handy}

Man kann Kontakten Profilfotos zuweisen, dann sieht man auch gleich wer anruft, ohne dass man den Namen lesen muss.
Wenn die Fotos auch noch einen einheitlichen Stil haben, umso besser.
Ich empfehle Mugshots in Schwarzweiß mit viel Kontrast.
Und natürlich müssen alle ein Schild mit ihrem Namen in der Hand halten.

%\chapter{Fensterln}
%\chapter{"romanum eunt domus"}

%\chapter{Kopierschutzmuster tätowieren lassen}
% TODO: Abbildung

%\chapter{Brief}
%\chapter{mobiler Gockerlbrater}
%\chapter{©Andi: Höhlenpiraten}

\chapter{Mit Fisch Gassi gehen}

Mit einem Hund Gassi gehen macht Spaß, man hat Bewegung an der frischen Luft und man kann sich mit anderen Hundebesitzern, die man auf dem Weg so trifft, über stets die gleichen Themen unterhalten. Aber was wirklich besonderes ist das natürlich nicht, außerdem macht ein Hund auch echt viel Arbeit.

Die Alternative:
Statt dem Hund führt man seinen Fisch aus!
Einfach das Aquarium auf einen kleinen Wagen stellen und um die Häuser ziehen - der Fisch kommt auch mal raus aus der Stube und die Street-Credibility steigt um Größenordnungen.
Sollte man in der Nähe eines Gewässers leben, kann man dem Fisch natürlich auch ein Geschirr mit Leine anlegen und ein bisschen die Promenade entlang spazieren - dann kann man sogar die Shisha rauchenden Jugendlichen anmaulen, dass sie doch kurz Platz machen mögen, weil man ja sonst nicht mit seinem Haustier vorbei kommt.

%\chapter{©Flo: Jesus-Serie}
%\chapter{Reusable Scientific Poster}
%\chapter{Gegensprechanlage im Labor}
%\chapter{Liste mit Sprachnachrichtakandidaten}
%\chapter{terminkalender für soziale kontakte}
%\chapter{biologischer geschirrspüler}
%\chapter{noreply emailaccount}

\chapter{Bikeathlon}

Biathlon ist eine etwas in die Jahre gekommene, aber durchaus interessante Sportart, bei der die Athleten mit Langlaufskiern im Kreis fahren und gelegentlich mit einem Gewehr auf Zielscheiben schießen.
An sich nicht schlecht, aber es geht nur im Winter und ein Gewehr hat auch niemand zuhause.
Und weil Sportarten grundsätzlich cooler werden, wenn sie von einem Fahrrad aus ausgeübt werden (zieht euch mal Bikepolo rein, total krank), hier die neue Idee: Bikeathlon.

Anstatt mit Langlaufskiern fahren die Athleten mit dem Rad (noch besser: Mountainbike) die Strecke entlang und auf die Ziele wird während der Fahrt mit Pfeil und Bogen geschossen.

%\chapter{slow messenger}
%\chapter{DateRate}
%\chapter{SocialPhone}
%\chapter{Drucker mit HDMI-Anschluss}
%\chapter{Dynamisches Gesellschaftsspiel}
%\chapter{paper tätowieren}
%\chapter{reviews/tutorials für viel zu spezifische sachen}
%\chapter{animal heatmap}
%\chapter{meme-explainer}
%\chapter{multimedia notizapp}
%\chapter{schreibmaschinengeräusche reverse engineeren}
%\chapter{demokratische realityshow}
%\chapter{dienstleister für ideenumsetzung}
%\chapter{abstrakte thumbnails}
%\chapter{songs am youtube thumbnail erraten}
%\chapter{schreibmaschine als drucker}
%\chapter{remote dating}
%\chapter{AR-Pinguine}
%\chapter{©RW: real life healthbar}
%\chapter{weitwinkelbrille}
%\chapter{vr reverse schaukelpferd}
%\chapter{real life inventory}
%\chapter{reverse engineering von straßen anhand eines rally-notizblocks}
%\chapter{stereoskopisches bewerbungsfoto}
%\chapter{erschwertes interview}
%\chapter{custom fake tattoo}
%\chapter{strg mit fuß drücken}
%\chapter{perkussion durch tanzbewegungen}
%\chapter{bias bei korrektur evaluieren}
%\chapter{air drumming synthesizer}
%\chapter{tierdokus über tiere, die gerade aus der narkose aufgewacht sind}
%\chapter{betrunkene rezitieren märchen}
%\chapter{icons mit machine learning aus skizzen generieren}
%\chapter{dynamisches gesellschaftsspiel}


\end{document}
