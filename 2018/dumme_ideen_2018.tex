% Template inspired by Micheal Müller's Book Template:
% https://www.overleaf.com/latex/templates/simple-a5-book-template/gkryymktcyzs

\documentclass[a5paper,pagesize,10pt,bibtotoc,pointlessnumbers,normalheadings,DIV=9,twoside=false]{scrbook}

\KOMAoptions{DIV=last}
\RedeclareSectionCommand[beforeskip=0pt,afterskip=5mm]{chapter}
\addtokomafont{chapter}{\large}

%\usepackage{trajan}
\usepackage{tgpagella}
\usepackage[T1]{fontenc}
\usepackage[utf8]{inputenc}
\usepackage[ngerman]{babel}
\usepackage{fontspec}

%\usepackage[babel,german=guillemets]{csquotes}

%\usepackage[sc]{mathpazo}
%\linespread{1.05} 

\usepackage{verbatim}
\usepackage{listings}

\usepackage{titlesec}
\usepackage{authblk}

\usepackage[hidelinks]{hyperref}

\usepackage{setspace}

\newcommand{\attribution}[1]{
    \textbf{Inspiration:} #1
}

%\setmainfont{TeX Gyre Pagella}

%\hyphenation{un-ver-packt}

%\setlength{\parindent}{10pt}
%\setlength{\parskip}{1.4ex plus 0.35ex minus 0.3ex}
%\setlength{\parskip}{1.4ex plus 0.35ex minus 0.3ex}

\title{Dumme Ideen 2018}   
\author{Andreas Schmid} 
\date{\today} 

\begin{document}

% Title page

\maketitle

% Inspirational quote thingy
\newpage{}
\thispagestyle {empty}

\vspace*{2cm}

\begin{center}
	\Large{\parbox{10cm}{
		\begin{raggedright}
		{\Large 
			\textit{Ein Buch ohne inspirierendes Zitat am Anfang lohnt sich nicht zu lesen.}
		}
	
		\vspace{.5cm}\hfill{---Andreas Schmid}
		\end{raggedright}
	}
}
\end{center}

\newpage

% toc

\singlespacing
\tableofcontents 

% Content!
\onehalfspacing

\chapter*{Vorwort}
Ich habe viele Ideen.
Seit 2018 schreibe ich sie auch auf.
Einige davon sind lustig, einige nützlich, einige peinlich - aber eins haben die meisten gemeinsam: Ich werde sie niemals umsetzen.
Deshalb hier eine stetig wachsende Sammlung von Ideen, macht damit was ihr wollt.
Aber wenn ihr eine davon umsetzt, gebt mir bitte Bescheid - das würde mich sehr freuen!

%\chapter{Rotes Telephon zwischen FIL und TechBase}

%\chapter{real live 3rd person steuerung (drohne + vr-brille)}
% TODO Skizze

%\chapter{tag-basierter musikrecommender}

\chapter{Kochsendung mit Werkzeug}

Man kennt sie, man liebt sie - Kochsendungen im Fernsehen oder im Internet.
Doch effektiv ist es doch immer das gleiche:
Ein leicht adipöser Herr mittleren Alters steht in einer viel zu gut ausgestatteten Küche, alle Zutaten sind in einer eigenen Schüssel, sodass man sich fragt, wer das ganze Zeug am Ende abspült und es wird zum hundertsten Mal irgend ein Bratensatz mit Rotwein abgelöscht, bevor wie von Zauberhand das bereits vorbereitete, fertige Gericht aus dem Ofen geholt wird.
Und auf keinen Fall das Lorbeerblatt vergessen!

Eine junge, von Killerspielen und Avengers-Filmen geprägte Zielgruppe spricht dieses Format wohl kaum an - was fehlt sind Action, Explosionen und Gefahr.
Deshalb die Idee: Warum nicht eine klassische Kochsendung etwas spannender gestalten, indem statt Kochmesser und Pürierstab Kreissäge und Bohrmaschine zum Einsatz kommen?
Dazu noch ein harter Heavy Metal-Sound"-track aus dem Makita-Baustellenradio und ein tätowierter Moderator, der ständig rumschreit und fertig ist die Kochsendung für die nächste Generation!

\chapter{Kochsendung mit Campingkocher}

Die Kochsendung mit Campingkocher stellt ein krasses Gegenstück zur letzten Idee dar.
Ein minimalistischer und auf das Nötigste beschränkter Lebensstil ist für die nachhaltige Hipster-Ge"-ne"-ra"-tion so wichtig wie der Fahrradparkplatz vor dem Un"-ver"-packt-Laden.
Eine kleine Japanerin zeigt uns, auf was wir eigentlich alles verzichten können, um ein glückliches Leben frei von Konsum und Kapitalismus zu führen.
Außerdem passt dann unser ganzes Zeug auch gut in den Dakine-Rucksack für den nächsten inspirierenden Trip nach Bali.

Da beim Backpacken das Kochen natürlich zu einer etwas größeren Herausforderung wird und wir uns nicht sicher sind, wie lange die gebratenen Nudeln beim Streetfood-Stand dieser älteren Dame nun wirklich schon im Wok vor sich hin brodeln, sind wir oft auf den Campingkocher angewiesen.
Dosenravioli sind natürlich auch nicht das wahre (Weißblech! Pfui!), deshalb brauchen wir eine Kochsendung, die nur mit dem Nötigsten auskommt:
Ein Campingkocher, ein Messer und alles, was uns Mutter Natur so bietet.
Und natürlich ein Hippie-Moderator mit beruhigender stimme, der uns zeigt, wie man leckere Gerichte mit dem Campingkocher zubereitet - damit wir auch auf Reisen nicht auf Nudeln mit Pesto verzichten müssen.
Amazing!

%\chapter{1. April: Besen anzünden}

%\chapter{Tutorialvideo: Besen bauen}

%\chapter{Wetteinsatz: Mit dem Radl aufs Summerbreeze fahren}

\chapter{Komplettes Videospiel als QR-code}
\attribution{Florian Bockes}

QR-Codes werden normalerweise verwendet, um Links zu Webseiten auf Papier zu drucken, Gegenstände mit IDs zu versehen und so weiter.
Aber in einem QR-Code kann man durchaus ein bisschen was an Daten unterbringen - 3 Kilobyte bei einem klassischen QR-Code, um genau zu sein.
Das ist zwar nicht wahnsinnig viel, aber vielleicht kann man ja mehrere kombinieren, um ein kleines Videospiel unterzubringen.
Vielleicht ein Textadventure?

Man würde dann einfach den QR-Code mit dem Smartphone scannen und schon beginnt das Spiel.
Ideal zum Beispiel für ein Wartezimmer, eine Schnitzeljagd, oder um zwischen den Spielen eine Rickroll zu verstecken.

%\chapter{Kalender mit falsch zugeordneten Zitaten}
% TODO: Abbildung

\chapter{Fotos von allen Kontakten am Handy}

Man kann Kontakten Profilfotos zuweisen, dann sieht man auch gleich wer anruft, ohne dass man den Namen lesen muss.
Wenn die Fotos auch noch einen einheitlichen Stil haben, umso besser.
Ich empfehle Mugshots in Schwarzweiß mit viel Kontrast.
Und natürlich müssen alle ein Schild mit ihrem Namen in der Hand halten.

%\chapter{Fensterln}

%\chapter{"romanum eunt domus"}

%\chapter{Kopierschutzmuster tätowieren lassen}
% TODO: Abbildung

\chapter{Brief schreiben}

Einfach mal oldshool jemandem einen Brief schreiben.
Damit rechnet heute niemand mehr.

\chapter{Mobiler Gockerlbrater}

Auf dem Campus der Universität Regensburg hat das Studentenwerk Niederbayern/Oberpfalz ein Monopol auf dem Verkauf von Lebensmitteln.
Leidtragend sind dabei die Studierenden, von denen sich viele eine Alternative zum Mensa- und Cafetenessen wünschen.
Der Wunsch nach einem Dönerstand auf dem Unicampus war sogar bereits Teil des Programms einer Hochschulpartei!

Da dies aufgrund des Monopols nicht so einfach möglich ist, kommt der mobile Gockerlbrater ins Spiel.
Normalerweise zu festen Wochentagen auf Supermarktparkplätzen anzutreffen, stellen insbesondere in ländlichen Gegenden die ``Grillhendl und Döner''-Wägen einen festen Bestandteil der unkomplizierten Nahrungsversorgung dar.
Dabei bleibt merkwürdigerweise der große Vorteil der Mobilität dieser Verkaufsstände gänzlich ungenutzt.
Durch Vorbestellung über eine Website und bargeldlosem Bezahlen könnte, ähnlich zu anderen Lieferdiensten, der Bedarf nach einem halben Hähnchen oder einem latschigen Döner kundgetan werden und das Fahrzeug macht sich unverzüglich auf den Weg zum Besteller.
Das Essen wird während der Fahrt zubereitet, um Zeit zu sparen.

So könnte auch auf dem Universitätsparkplatz problemlos Essen verkauft werden, ohne Probleme mit dem Studentenwerk zu bekommen - denn bis die das mitbekommen, ist man schon längst über alle Berge.

%\chapter{©Andi: Höhlenpiraten}

\chapter{Mit Fisch Gassi gehen}

Mit einem Hund Gassi gehen macht Spaß, man hat Bewegung an der frischen Luft und man kann sich mit anderen Hundebesitzern, die man auf dem Weg so trifft, über stets die gleichen Themen unterhalten. Aber was wirklich Besonderes ist das natürlich nicht, außerdem macht ein Hund auch echt viel Arbeit.

Die Alternative:
Statt dem Hund führt man seinen Fisch aus!
Einfach das Aquarium auf einen kleinen Wagen stellen und um die Häuser ziehen - der Fisch kommt auch mal raus aus der Stube und die Street-Credibility steigt um Größenordnungen.
Sollte man in der Nähe eines Gewässers leben, kann man dem Fisch natürlich auch ein Geschirr mit Leine anlegen und ein bisschen die Promenade entlang spazieren - dann kann man sogar die Shisha rauchenden Jugendlichen anmaulen, dass sie doch kurz Platz machen mögen, weil man ja sonst nicht mit seinem Haustier vorbei kommt.

%\chapter{©Flo: Jesus-Serie}

\chapter{Reusable Scientific Poster}
% TODO: Abbildung

Auf wissenschaftlichen Konferenzen ist es üblich, Kurzbeiträge und laufende Projekte in ``Postersessions'' zu präsentieren.
Dabei werden zig Poster im gigantischen A0-Format an Pinwänden in einem Saal ausgestellt, die Autoren der Beiträge befinden sich in der unmittelbaren Nähe und erzählen Leuten, die sich entweder wirklich für das Projekt interessieren, sowie denen, die gerade eine halbe Stunde bis zum nächsten Vortrag überbrücken müssen, immer wieder das gleiche.

Das Erstellen dieser Poster ist eine Kunst für sich:
Man kämpft stundenlang mit einem Layoutprogramm, mit dem man sich kaum auskennt und versucht, den Inhalt der Arbeit irgendwie auf der doch recht knapp bemessenen Fläche unterzubringen und dabei das Corporate Design der Uni nicht komplett mit Füßen zu treten.
Dabei zerschießt der automatische Blocksatz regelmäßig das komplette Layout, weswegen man sich mit der Zeit ein beeindruckendes Arsenal an verschieden langen Synonymen für häufig benutzte Wörter zulegt.

Wenn man es dann endlich geschafft hat, das Poster fertigzustellen und im besten Fall auch noch jemand drübergelesen hat, ist es so weit:
Man druckt das Ungetüm über das verwirrende Webinterface der Unidruckerei aus, hofft, dass man tatsächlich A0 ausgewählt hat (und nicht 12€ für eine A4-Seite auf A0-Hochglanzpapier bezahlt) und holt es dann im zusammengerollten Zustand ab.

Doch egal wie oft man selbst oder andere drübergelesen haben, es scheint ein Naturgesetz zu sein, dass man direkt beim ersten Betrachten des gedruckten Meisterwerks einen Tippfehler in der 15 cm hohen Überschrift findet.
Dann kann man nur hoffen, dass die Druckerei noch offen hat, denn der Zug zur Konferenz geht in diesem Fall eigentlich immer früh morgens am nächsten Tag.

Ein digitales Poster (zum Beispiel ein großes ePaper-Display) wäre eine Möglichkeit, die genannten Probleme zu lösen.
Der Inhalt des Posters kann notfalls auch vor Ort auf der Konferenz geändert werden, falls jemand bemerkt, dass die abgebildeten Messwerte nicht stimmen.
Die Anschaffungskosten werden innerhalb kürzester Zeit durch die gesparten Druckkosten gedeckt und weniger Papiermüll ist es auch.


%\chapter{Gegensprechanlage im Labor}

\chapter{Liste mit Sprachnachrichtskontakten}

Die Sprachnachricht ist ein interessantes kulturelles Phänomen der Generation WhatsApp.
Sie vereint das Schlechte beider Welten aus Telephonie und Sofortnachrichten und kann zur asynchronen Kommunikation verwendet werden.
Das tolle für den Sender (und schlechte für den Empfänger) der Nachricht: Man weiß vorher nicht, auf was man sich einlässt, wenn man die Nachricht anhört.

Dies macht die Sprachnachricht zu einem großartigen Werkzeug, fachkundige Menschen nach ihrer Expertise zu fragen.
Betreibt man diese Art der Informationsaquise systematisch, so lohnt es sich, eine Liste mit potentiellen Ansprechpartnern für jedes Themengebiet zur Hand zu haben.

% TODO: Liste mit empfohlenen Fachgebieten
%Nützliche Kontakte sind unter anderem:
%\begin{itemize}
%    \item Juristen
%    \item Ingenieure
%    \item Biologen
%    \item Physiker
%    \item Köche, Bäcker, Konditoren
%\end{itemize}

%\chapter{terminkalender für soziale kontakte}

%\chapter{biologischer geschirrspüler}
% TODO Abbildung

\chapter{Noreply E-Mail-Account}

Informationen per E-Mail, die keiner Antwort bedürfen (beispielsweise Bestellbestätigungen, Zahlungserinnerungen, etc.) werden oft automatisch mit einer so genannten ``noreply''-Adresse versandt (Format: \emph{noreply@irgendein-shop.de}).
Zusätzlich zum doch recht eindeutigen Namen der Adresse befindet sich meist noch ein Hinweis im Text der E-Mail, dass man auf diese Mail doch bitte \emph{nicht} antworten soll.

Gerade deshalb wäre es interessant herauszufinden, was die spezielle Gruppe an Menschen, die auf solche E-Mails antwortet, denn so zu erzählen hat.
Technisch sollte dies recht einfach lösbar sein, indem man die noreply-Adresse mit einem Postfach koppelt und hin und wieder reinschaut, was da so drinsteht.
Ein guter Spamfilter ist in diesem Kontext wohl fast obligatorisch. 

\chapter{Bikeathlon}

Biathlon ist eine etwas in die Jahre gekommene, aber durchaus interessante Sportart, bei der die Athleten mit Langlaufskiern im Kreis fahren und gelegentlich mit einem Gewehr auf Zielscheiben schießen.
An sich nicht schlecht, aber es geht nur im Winter und ein Gewehr hat auch niemand zuhause.
Und weil Sportarten grundsätzlich cooler werden, wenn sie von einem Fahrrad aus ausgeübt werden (zieht euch mal Bikepolo rein, total krank), hier die neue Idee: Bikeathlon.

Anstatt mit Langlaufskiern fahren die Athleten mit dem Rad (noch besser: Mountainbike) die Strecke entlang und auf die Ziele wird während der Fahrt mit Pfeil und Bogen geschossen.

%\chapter{slow messenger}

%\chapter{DateRate}

%\chapter{SocialPhone}
% drängt dich dazu, zurückzuschreiben

\chapter{Drucker mit HDMI-Anschluss}

Während im goldenen Zeitalter der Computer noch fleißig Lochkarten gestanzt wurden und ein bisschen später Endlospapier durch die Matrixdrucker ratterte, ist der Drucker relativ bald als bevorzugtes Ausgabegerät vom echtzeitfähigen Bildschirm abgelöst worden.

Doch auch wenn es ein bisschen langsamer und ökologisch fragwürdig ist:
Ein Drucker ist prinzipiell in der Lage, graphische Nutzeroberflächen, Webseiten und sogar Einzelbilder von Videos darzustellen.
Warum also diese Funktionalität komplett vom Nutzer kapseln, wenn man die Geräte ohne größeren Aufwand mit einem HDMI-Eingang versehen könnte?

Mit bis zu 100 Seiten pro Minute schaffen Profigeräte\footnote{\url{https://www.brother-usa.com/products/hls7000dn}} auch Framerates, die ich in meiner Jugend bei Videospielen als durchaus vertretbar empfand.

%\chapter{Dynamisches Gesellschaftsspiel}
%\chapter{paper tätowieren}
%\chapter{reviews/tutorials für viel zu spezifische sachen}
%\chapter{animal heatmap}
%\chapter{meme-explainer}
%\chapter{multimedia notizapp}

\chapter{Schreibmaschinengeräusche Reverse-Engineeren}

Mechanische Tastaturen erfreuen sich größter Beliebtheit.
Angeblich liegt das natürlich daran, dass die Haptik des Tastenanschlags so angenehm ist und so weiter, aber eigentlich geht es doch nur darum, der lauteste Tipper im Großraumbüro zu sein:
Mit maschinengewehartigen Salven werden die Kollegen subtil darauf aufmerksam gemacht, dass man gerade hart am Schuften ist.

Doch möglicherweise ist das dem Hipster von heute noch nicht genug?
Während das martialische Rattern der grünen Cherry-Switches bei Gamern und IT-Leuten vielleicht nostalgische Erinnerungen an die Jugend auf de\_dust2 auslöst und sie dadurch in ihrem täglichen Dienst für die Tech-Industrie zu neuen Höchstleistungen anspornt, steht doch kein Geräusch so sehr für kreatives Schaffen wie das monoton mechanische Klicken einer guten alten Schreibmaschine.

Doch da die Kreativen von heute verständlicherweise nicht mehr auf Retina-Display und Thunderbolt-HDMI-Adapter verzichten können, sind sie an eine winzige Laptoptastatur mit gefühlt einem halben Millimeter Anschlag gebunden.
Um sich beim Tippen des eigenen Fitnessblogs trotzdem wie der nächste Mark Twain zu fühlen, könnte der Computer während des Tippens gesampelte Schreibmaschinengeräusche abspielen - inklusive dem ``Ratsch!'' beim Drücken der Eingabetaste, versteht sich.

%\chapter{demokratische realityshow}
%\chapter{dienstleister für ideenumsetzung}
%\chapter{abstrakte thumbnails}

%\chapter{songs am youtube thumbnail erraten}
% TODO Abbildung

%\chapter{schreibmaschine als drucker}
%\chapter{remote dating}

%\chapter{AR-Pinguine}
% das war irgendwas mit sex glaub ich...

%\chapter{©RW: real life healthbar}

\chapter{Weitwinkelbrille}

Brillen und Kontaktlinsen werden normalerweise dafür benutzt, um Sehschwächen auszugleichen, beispielsweise indem eine verschobene Fokusebene so korrigiert wird, dass sie wieder auf der Netzhaut liegt.
Doch Optik hat noch viel mehr auf dem Kasten, als nur Bilder scharf zu stellen!
Durch die richtige Kombination von Linsen kann das Sichtfeld auch verengt (Tele) oder erweitert (Weitwinkel) werden.
Wieso nutzt man also nicht diese in der Photographie weit verbreitete Technik, um spezielle Weitwinkelbrillen zu bauen?
Dadurch könnte man deutlich weiter zur Seite sehen, ohne den Kopf zu bewegen, was zum Beispiel im Straßenverkehr nützlich sein könnte.
Außerdem bekommen alle eine total große Nase, wenn man mit einer Weitwinkeloptik nah genug ans Gesicht geht!

%\chapter{vr reverse schaukelpferd}
%\chapter{real life inventory}
%\chapter{reverse engineering von straßen anhand eines rally-notizblocks}

\chapter{Stereoskopisches Bewerbungsfoto}
% TODO Abbildung

Wer schon mal eine Bewerbung gesehen hat weiß, dass nichts der Realität ferner ist, als das gezwungen aus dem Konfirmationsanzug herausgrinsende Etwas in der rechten oberen Ecke.
Die 15€, die vor einem Jahrzehnt beim Dorfphotographen in vier Abzüge der bis ins künstlich Glatte retouchierten Visage investiert wurden, sollen sich schließlich lohnen!

Während ein verwackeltes Frontkamerabild vom letzten Samstag der Realität im Normalfall deutlich näher kommt, besitzt aber wohl niemand die Ehrlichkeit, dieses als ersten Eindruck mit einer Bewerbung mitzuschicken.
Doch glücklicherweise gibt es eine fortschrittliche Technologie aus der Wackelbildindustrie, durch die man das Beste beider Welten vereinen kann:
Stereoskopische Bilder!

Das Prinzip ist ganz einfach:
Man nehme eine Oberfläche mit ganz vielen kleinen, spitzen Rillen.
Beide Bilder werden in sehr dünne Streifen geschnitten.
Die Streifen von Bild A werden auf die eine, die Streifen von Bild B auf die andere Seite der Rillen geklebt - ein Wackelbild halt.

Dadurch können zwei Bilder in einem Bewerbungsfoto untergebracht werden und je nach Betrachtungswinkel sieht man entweder den Abiturienten im Anzug oder ein zombieartiges Wesen, das gerade Bier aus einem Trichter trinkt.

\chapter{Erschwertes Interview}

Interviews in Form eines ausführlichen Dialogs können spannende Einblicke in das Leben, die Geschichte und die Ansichten von Vorbildern und Idolen geben.
Doch je prominenter ein Mensch, desto mehr solcher Interviews müssen geführt werden und entsprechend wird oft mit Standardsätzen oder Wischi-Waschi-Aussagen auf spannende Fragen geantwortet.

Um ehrliche Antworten auf tiefer gehende Fragen aus Interviewpartnern herauszukitzeln und zu sehen, wie sie sich ohne die aufgesetzte Maske der Professionalität verhalten, sind extreme Maßnahmen notwendig.
Begeben sich Interviewer und Interviewter gemeinsam an ihr körperliche und/oder psychische Grenze, so kommen möglicherweise Aussagen und Persönlichkeitszüge zum Vorschein, die in einer gewöhnlichen Interviewsituation eher unterdrückt werden.

Man könnte mit dem Interviewpartner beispielsweise Achterbahn fahren, klettern, die Hand auf eine langsam heißer werdende Herdplatte legen, Drogen nehmen oder Sport machen - und währenddessen Fragen stellen.


\chapter{Custom Fake-Tattoo}

Eine Tätowierung ist eine bewusste Entscheidung für eine permanente Veränderung des eigenen Körpers, ohne zu wissen, ob sie dem Zukunfts-Selbst nicht vielleicht Probleme machen könnte.
Das ist riskant und selbstbewusst - und deshalb auch echt cool.
Allerdings kann das Ganze natürlich auch nach hinten losgehen, zum Beispiel wenn man im Nachhinein feststellt, dass es einen Grund gibt, warum der Tätwierer so günstig war, oder wenn einem nach einem Jahr der damals noch so coole japanische Drache auf der Schulter eher peinlich ist.

Während ich dem besoffenen Punk aus der Stammkneipe leider keine künstlerischen Fähigkeiten verleihen kann, habe ich zumindest für das andere Problem eine Lösung.
Die vielen bestimmt noch aus dem Mickey-Maus-Prospekt bekannten Fake-Tattoos lassen sich ohne die Hilfe eines Tätowierers auf dem eigenen Körper platzieren und mit einer Wurzelbürste und etwas guten Willen leicht wieder entfernen.
Gäbe es diese nicht nur mit Donald Duck, sondern als Sonderanfertigungen mit eigenen Motiven, könnte man einfach so ein Tattoo mit dem Motiv seiner Wahl bestellen und erstmal ausprobieren, bevor man sich den Oberschenkel für den Rest des Lebens mit einem Leopardenmuster verschönern lässt.

%\chapter{strg mit fuß drücken}
% gibts schon, TODO Quelle

%\chapter{perkussion durch tanzbewegungen}

%\chapter{bias bei korrektur evaluieren}

%\chapter{air drumming synthesizer}

\chapter{Tierdokus über Tiere, die gerade aus der Narkose aufgewacht sind}

Ähnlich wie Menschen sind Tiere nach einer Narkose relativ dumm und machen entsprechend lustige Sachen.
Sie fallen zum Beispiel um, laufen gegen Türrahmen und so weiter.
Doch anders als beim Menschen ist es gesellschaftlich relativ gut akzeptiert, sich über Tiere lustig zu machen.
Als zusätzliches Genre an lustigen Tiervideos schlage ich deshalb Videos von Tieren vor, die gerade aus der Narkose aufgewacht sind.

\chapter{Betrunkene rezitieren Märchen}

Die Gebrüder Grimm stellen eine feste Instanz in der deutschen Literaturgeschichte dar - so durften sie doch sogar den 1000 Mark Schein zieren!
Deshalb ist es auch kein Wunder, dass die meisten Anhänger unseres Kulturkreises mit diversen Märchen aufgewachsen sind.
Doch von der Kindheit bis zu dem Punkt, an dem man sich aufgrund des eigenen Nachwuchses zwangsweise erneut damit befasst, vergeht meist eine durchaus lange Zeit, in der man mit Märchen eher wenig am Hut hat.
So kann es schonmal passieren, dass man den genauen Verlauf der Geschichten nicht mehr hundertprozentig parat hat, einige Details vergessen werden oder gar zwei voneinander völlig unabhängige Märchen im Geiste miteinander vereint.

Und auch wenn es gewiss schon ganz unterhaltsam wäre, junge Erwachsene ein Märchen aus dem Gedächtnis rezitieren zu lassen, so steigt der Spaß enorm, wenn diese dabei auch noch alkoholisiert sind.
Noch besser könnte es werden, wenn mehrere Betrunkene beim Zusammensetzen der Geschichten aus losen Erinnerungen kollaborieren.

%\chapter{icons mit machine learning aus skizzen generieren}

\chapter{Dynamisches Gesellschaftsspiel}

Ein geselliger Abend mit anderen Leuten kann ganz nett sein, aber manchmal kommt einfach nicht so wirklich Stimmung auf.
Ob es nun daran liegt, dass man sich noch nicht so gut kennt und deshalb nicht weiß, was man reden soll, oder daran, dass man sich seit Jahren dreimal die Woche in der selben Kneipe trifft und einfach schon alles gesagt ist - ein letzter mutmaßlicher Strohhalm sind Gesellschaftsspiele, mit denen man die Zeit irgendwie rumbringt, ohne sich wirklich über etwas unterhalten zu müssen.
Nun kann es natürlich sein, dass man gerade zufällig kein komplettes Ravensburger-Sortiment mit sich führt, man jedes Spiel schon kennt, oder sich einfach nicht mit den Anderen auf ein bestimmtes Spiel einigen kann.
Doch das ist nicht schlimm, denn jetzt gibt es das dynamische Gesellschaftsspiel!

Die Regeln sind ganz einfach: Es gibt keine - zumindest noch nicht.
Alle Mitspieler (mindestens drei, je mehr umso komplizierter, am besten betrunken) denken sich die Regeln eines Spiels reihum aus.
Der erste Spieler beschreibt die Ausgangssituation, jeder weitere Spieler führt eine Regel oder Bedingung ein und der letzte Spieler gibt schließlich die Siegesbedingung bekannt.
Dann wird das Spiel, das man sich gerade ausgedacht hat, gespielt und danach beginnt das Ganze mit allen Überlebenden von vorne.

\emph{Ich hab das schon mehrmals ausprobiert und mit den richtigen Leuten macht das wirklich Spaß. Könnte mir auch vorstellen, dass es für so komische Teambildungsworkshops geeignet wäre - aber falls das jemand macht, lasst mich da bitte aus dem Spiel.}

\end{document}
